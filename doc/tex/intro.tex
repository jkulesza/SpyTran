%-----------------------------------------------------------------------------%

Commonly when dealing with the spatial discretization of the transport equation
an explicit finite differencing scheme is employed. An explicit differencing
formulation allows one to ``sweep'' through the mesh computing the flux by utilizing
previously computed flux values (or boundary values) in the trailing nodes to
project the flux to the node ahead.  In computation time this strategy scales
linearly with the number of nodes. This is predominately why differencing
schemes are still commonly employed to resolve the spatial dependence of the
flux in the transport equation. While finite differencing is ideal for 1D
geometries, in multidimensional problems differencing schemes are difficult to
extend to irregular meshes. The simplest codes are limited to rectilinear
meshes, representing the geometry with squares and rectangular regions only.
This leads to poor capture of curved or organically shaped edges unless one
very finely meshes the regions around such geometric features. It is possible
to construct a differencing scheme for arbitrary non rectilinear meshes, but it
would require pre-computing a “sweep map” which determines the node order in
which the mesh is swept such that the stability of the method is preserved. In
addition the differencing equations themselves become slightly more cumbersome
to generate as the nodes are no longer at right angles to each other.

Other spatial discretization techniques are well suited for irregular meshes. Common
strategies for discretizing PDE’s on irregular meshes are the finite volume and
the finite element techniques. We will focus on the finite element method in
this work. Specifically, this work involves the application of discontinuous
finite elements to the linear, first order radiation transport equation.

The Discontinuous Galerkin (DG) method was introduced by Reed and Hill (1973)
to approximate the solution of hyperbolic PDEs \cite{lesaint}. One of the first applications of the
method was to the radiation transport equation \cite{reed}. The
DG method was later extended to elliptic and parabolic equations \cite{riviere}.  Today,
DG methods are extraordinarily pervasive in the numerical and applied mathematics
communities.  DG's popularity is due to the compact nature of the discretization scheme allowing for excellent
scaling on many CPUs and due to ease of increasing the order of accuracy of the
solution approximation over an element.  It also allows one to tailor the coupling between
neighboring elements by adjusting the so-called numerical flux which conveys information
about the transport of the conserved quantity across cell boundaries.

Since the solution approximation afforded by the DG method does not enforce continuity at
element boundaries it is possible to capture sharp discontinuities in the field
of interest naturally.  This property is important for radiation transport problems
in which the flux jumps across extremely thin highly absorptive coatings;
Such is the case in simulations of IFBA fuel.

It is also worth mentioning that this work is an extension of the methods introduced in
the computational methods in radiation transport graduate course offered at the University
of Texas at Austin.  The theory and consequences of applying the discrete ordinate treatment to the angular
dependence and the multigroup approximation of the energy dependence is left to the
bulk of the course material.
The text: Computational Methods of Neutron Transport by E.E. Lewis \cite{Lewis}
covers these basics in detail as well.
